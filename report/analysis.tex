%%%%%%%%%%%%%%%%%%%%%%%%%%%%%%%%%%%%%%%%%%%%%%%%%%%%%%%%%%%%%%%%%%%%%%%%%%%%%%%%
\chapter{Анализ предметной области}
%%%%%%%%%%%%%%%%%%%%%%%%%%%%%%%%%%%%%%%%%%%%%%%%%%%%%%%%%%%%%%%%%%%%%%%%%%%%%%%%

\section{Анализ актуальности}
В настоящее время к скорости распространения информации предъевляются высокие требования. Это обусловлено тем, что возникает необходимость быстро реагировать на какие-либо изменения и корректировать план своей деятельности. Когда закрывается одна из станций метро, все пассажиры получают об этом своевременное оповещение и выбирут более удобный маршрут. В структуре ВУЗа аналогичным примером является изменеие расписания занятий. Задача обзвона и личного уведомления каждого студента занимает дляительное время и требует много человеческих ресурсов. Специальный сервис с расписанием, доступ к которому есть у каждого студента, решает эту проблему в считанные минуты.

Однако, чаще всего необходимо получить доступ не к общедоступной информации, а к персональным данным. Это могут быть сведения о заказе в интернет-магазине, о задолженности в библиотеке, об оплате проживания или штрафов. У этой категории задач также есть два пути решения. Первый - консервативный. вы можете лично позвонить или придти по нужному адресу и узнать все, что вам нужно. Но, во-первых, это затратно по времени, силам и, возможно, средствам. Во-вторых, несмотря на все вложения данный алгоритм действий не гарантирует получение результата в любое время. Так в магазине может быть выходной, а у диспетчера телефонной службы слишком много звонков. Второй путь - современный. Для этого банки, магазины, библиотеки и другие организации создают электронные личные кабинеты. Такая система не только решает проблему быстрого доступа к персональным данным, но и позволяет пользователю удаленно совершать определенные действия, такие как заказ квитанций в банке или книг в библиотеке.

С аналогичными проблемами сталкивается студент высшего учебного заведения. Ему также приходится работать как с общедоступной информацией (расписание занятий, новости о различных событиях, презентации  и матириалы курсов), так и с персональными данными (сведения об успеваемости, оплате обучения и проживания в общежитии, заявках на повышенную стипендию и участие в олимпиадах). Таким образом, создание электронной системы, которая значительно упрощает выполнение описанных выше действий - это актуальная задача для каждого высшего учебного заведения в России и в мире.

Стоит учитывать, что при создании такой электронной системы возникает еще одна актуальная проблема. Это проблема доступности сервиса на различный устройствах. Данная проблема особенно важна в контексте создания личного кабинета студента. Это обусловлено тем, что молодое поколение при необходимости получить ту или иную информацию чаще обращается к мобильному устройству, а не к компьютеру. Поэтому необходимо учитывать данный аспект при разработке системы. Например, создание мобильной версии сайта или мобильного приложения решает данную проблему.

\section{Существующие решения}

	\subsection{Мобильное приложение <<Расписание ВУЗов>>}
	
	\subsection{Образовательная сеть <<Дневник.ру>>}
	
	\subsection{Личный кабинет обучающегося СПбГУ}
	
	\subsection{Личный кабинет студента СПбГУТ}
	
	\subsection{Личный кабинет студента СПбПУ}

\section{Сравнительный анализ}

\section{Постановка задачи}